\chapter*{Kurzfassung}
\thispagestyle{empty}

Diese Arbeit beschreibt die Entwicklung des ersten open-source
VHDL-Frontends mit dem Namen Yodl. Yodl selbst benutzt das quelloffene
Synthesetoolkit Yosys als Basis.

Zunächst wird kurz der aktuelle Status der Entwicklung digitaler
Schaltungen reflektiert. Außerdem wird Yosys kurz beschrieben.
Danach beschreibt die Arbeit erst generelle,
compiler-spezifische Probleme, die gelöst werden müssen. Sodann geht
sie detailliert auf die VHDL-bezogenen Probleme ein und zeigt die
entwickelten Algorithmen und Design\-entscheidungen auf. Schließlich
liefert diese Masterarbeit einen Ausblick in die Zukunft von Yodl,
wobei hier
sowohl die momentanen Limitierungen der Implementierung als auch
Verbesserungen des Testsystems betrachtet werden.

Diese Abschlussarbeit dokumentiert und erklärt den Quellcode von
Yodl. Dieser liegt der Arbeit bei und ist gleichzeitig das wichtigste
Ergebnis.

Interessant ist sowohl der schriftliche als auch der Quellcodeteil
dieser Ausarbeitung für Hardwareentwickler, VHDL-Programmierer,
Yosys-Nutzer und Theoretiker aus dem Compilerbau und dem Gebiet der
formalen Sprachen. Insbesondere für Theoretiker ist der praktische
Aspekt dieser Arbeit relevant.

\bigskip

\noindent
Schlüsselworte: Compiler, Formale Sprachen, VHDL, Yosys, Hardwaresynthese